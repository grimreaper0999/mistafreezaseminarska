\extrachap{Predgovor}

S povečevanjem gostote integriranih vezij postajajo tranzistorji vse manjše nanoelektronske naprave. Želeno delovanje tranzistorjev velikosti le nekaj nanometrov pa pričnejo ovirati kvantni pojavi in sile med elektroni. Zato so se pojavile zamisli o novih platformah, ki jih ti pojavi ne le ne bi ovirali, pač pa bi jih platforme izkoristile za procesiranje. Poleg tega lahko nove nanoelektronske naprave prinesejo še več prednosti pred današnjimi računalniki, npr. večjo gostoto, manjšo porabo energije, hitrejše delovanje itd.


Ena od takih predlaganih platform so kvantni celični avtomati (Quantum Cellular Automata - QCA). Namesto tranzistorjev vsebujejo celice z elektroni, katerih razporeditev znotraj celice določa logično stanje. Procesiranje se izvede z ustrezno razporeditvijo QCA celic, pri čemer se izkoriščata Coulombova sila in kvantno tuneliranje. Do sedaj so bile fizično implementirane le manjše QCA strukture z majhnim številom celic, ker je praktično težko izdelati QCA celice nanometrske velikosti. Možno pa je načrtovati in optimizirati večja QCA vezja s pomočjo računalniškega simulatorja.


V zadnjem času je zaradi praktične implementacije veliko pozornosti posvečene kvantnemu procesiranju z osnovnimi gradniki kvantnimi biti. Superpozicija in kvantna prepletenost omogočata novo paradigmo procesiranja, ki obljublja eksponentno pohitritev nekaterih algoritmov glede na klasične računalnike. Izgradnjo skalabilnega kvantnega računalnika trenutno otežuje zahteven nadzor neizoliranega sistema kvantnih bitov. Vendar predstavljene implementacije, množica razvitih kvantnih algoritmov in protokolov, dostopni računalniški simulatorji in možnost uporabe manjših fizičnih kvantnih računalnikov preko spleta dopuščajo preučevanje in razvoj področja.


Tako QCA kot kvantnemu procesiranju je skupna reverzibilnost. Fizično reverzibilno procesiranje brez izgube informacij se približa minimalni možni porabi energije. Kvantno procesiranje je inherentno reverzibilno, saj informacije v kvantnem bitu ni mogoče izbrisati. Primerna zasnova QCA vezja omogoča implementacijo fizično reverzibilne procesne platforme.